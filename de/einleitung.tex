% stura-voting-doc (c) by Fabian Wenzelmann
%
% stura-voting-doc is licensed under a
% Creative Commons Attribution 4.0 International License.
%
% You should have received a copy of the license along with this
% work. If not, see <http://creativecommons.org/licenses/by/4.0/>.

\chapter{Einleitung}
Das StuRa-Abstimmungstool wird benutzt um die Abstimmungen des StuRa
auszuwerten.
Das Tool unterstützt das \href{https://de.wikipedia.org/wiki/Median}{Median-Verfahren}
und die \href{https://de.wikipedia.org/wiki/Schulze-Methode}{Schulze-Methode}.
Zielgruppe sind Personen, die mit der Software arbeiten müssen, vor allem das
StuRa-Präsidium welches die Auszählung durchführt.

\section*{Software}
Die Software ist als Open Source Projekt unter der \href{http://www.apache.org/licenses/LICENSE-2.0}{Apache Lizenz, Version 2.0}
verfügbar.
Der Code des Projekts mit weiteren Informationen zur Software findet sich auf
\href{https://github.com/FabianWe/stura-voting-django}{github.com}.

\section*{Hinweise}
Bei der Software handelt es sich um ein \emph{Hilfsmittel} zum Auszählen.
Beschlüsse werden nur vom Präsidium des StuRa veröffentlicht.
Alle Abstimmungsergebnisse sollten auf jeden Fall noch einmal überprüft werden,
die Ausgaben der Software sind nicht verbindlich!
Es gilt: Wann immer jemand etwas einträgt / editiert / auswertet sollten die
Daten nochmal überprüft werden!

Dieses Handbuch erklärt alle Funktionen des Tools, nicht jede*r Nutzer*in kann
all diese Aktionen auch durchführen.
Es hängt davon ab, welche Rechte einem Account zugeteilt wurden.
